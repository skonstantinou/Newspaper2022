%\begin{MyArticle}[enhanced, height=0.2\textheight,
%tikz={rotate=0}]{Physicists Find Elusive Particle Seen as Key to
%Universe}
\begin{MyArticle}[enhanced, tikz={rotate=0},
    width=1.0\textwidth]{\huge Physicists Find Elusive Particle Seen as Key to Universe}
  \begin{multicols}{2}
    Results are presented from searches for the standard model Higgs
    boson in proton–proton collisions at and 8 TeV in the Compact Muon
    Solenoid experiment at the LHC, using data samples corresponding
    to integrated luminosities of up to 5.1 fb$^{−1}$ at 7 TeV and 5.3 fb$^{−1}$
    at 8 TeV. The search is performed in five decay modes:
    $\gamma\gamma$, $ZZ$,  $\tau^{+}\tau^{-}$, and $b\bar{b}$.
    An excess of events is observed above the expected background,
    with a local significance of 5.0 standard deviations, at a mass
    near 125 GeV, signalling the production of a new particle. The
    expected significance for a standard model Higgs boson of that
    mass is 5.8 standard deviations. The excess is most significant in
    the two decay modes with the best mass resolution, $\gamma\gamma$ and $ZZ$; a
    fit to these signals gives a mass of 
    $125.3\pm0.4(\text{stat.})\pm0.5(\text{syst.})$ GeV. The decay to
    two photons indicates that the new particle is a boson with spin 
    different from one. 
    % ========================
    \begin{figure}
      \begin{center}
        \vspace{-0.2in}
        \leavevmode
        \includegraphics[width=0.5\textwidth]{./figures/HiggsBosonDiscoveryBW.png}
      \end{center}
    \end{figure}
    % ========================
  \end{multicols}
\end{MyArticle}
